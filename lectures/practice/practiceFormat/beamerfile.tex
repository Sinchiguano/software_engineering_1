\documentclass{beamer}



\usetheme[progressbar=frametitle]{metropolis}
%\usecolortheme{crane}
\setbeamertemplate{frame numbering}[fraction]
\useoutertheme{metropolis}
\useinnertheme{metropolis}
\usecolortheme{spruce}
\setbeamercolor{background canvas}{bg=white}


\definecolor{mygreen}{rgb}{.125,.5,.25}
\usecolortheme[named=mygreen]{structure}



\title{NOTACION CIENTIFICA}
\author{SINCHIGUANO CESAR}
\institute{ULEAM}
\date{\today}



%%%%%%%%%%%%%%%%%%%
\begin{document}
\metroset{block=fill}

\maketitle


\begin{frame}[t]{NOTACION}\vspace{10pt}
\begin{enumerate}
\item POTENCIA 10

\item POTENCIA 11
\end{enumerate}
Hello
\end{frame}




\begin{frame}[t]{NOTACION}\vspace{10pt}

\end{frame}



\begin{frame}[t]{NOTACION}\vspace{4pt}
\begin{block}{EXPRESION CIENTIFICA}
\vspace{0.5em}
Notación científica. Notación de ingeniería

La notación científica. Notación de ingeniería permiten representar cantidades muy grandes o muy pequeñas, muy comunes en áreas de la tecnología como la electricidad y electrónica, entre otras.




\vspace{0.5em}


\end{block}	
\vspace{10pt}
Set \line(1,0){50} lajdklfj . \\[10pt]
	
\end{frame}




\end{document}